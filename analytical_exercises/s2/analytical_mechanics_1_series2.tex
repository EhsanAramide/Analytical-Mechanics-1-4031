%!LW recipe=latexmk (xelatex)

% See `LICENCE' file in root directory for more information
% Only use XeLaTeX compiler 

\documentclass{article}

% Packages
\usepackage{fancyhdr}
\usepackage[a4paper, 
top=1in,
marginparwidth=0.6in,
left=1in, right=0.75in]
{geometry}

\usepackage[headings=runin, 
            skip-below={1\baselineskip},
            counter-format={qu[1]-}]
            {exsheets}
\usepackage{enumitem}
\usepackage{amsmath}

\usepackage{xepersian}

% Packages initializations
% `xepersian'
\settextfont[Scale=1]{XB Zar}
\setlatintextfont[Scale=1]{Junicode}
\setdigitfont{XB Zar}

% `fancyhdr'
\pagestyle{fancy}
\fancyhf[HR]{سید امیر رضا طالقانی / احسان آرمیده}
\fancyhf[HL]{مکانیک تحلیلی ۱ / تمرین سری ۲}
\fancyhf[FC]{\thepage}

\linespread{2}
\SetupExSheets[question]{name=}


\begin{document}
\noindent
\section*{سوالات}
از فصل ۲ کتاب مکانیک تحلیلی (فولز ویراست ۷ام) سوالات روبه‌رو حل بشود:
(فقط مورد «ب» سوالات ۱، ۲، ۳) / ۱۰ / (۱۲، ۱۳) / ۱۴
\\

\noindent
{\large سوالات زیر نیز حل بشوند:}
    \begin{question}
        اگر یک پرتابه، از مرکز دستگاه مختصات، با سرعت اولیه‌ی $v_i$
        به صورتی پرتاب شود که با افق زاویه‌ی $\alpha$
        داشته باشد، آنگاه زمان مورد نیاز برای آنکه پرتابه از خط گذرا از مبدأ L
        (که زاویه‌ی $\beta$ با افق دارد، به صورتی که $\beta < \alpha$ است)
        عبور کند را محاسبه کنید.
    \end{question}

    \begin{question}
        یک پرتابه با سرعت اولیه‌ی $v_i$ پرتاب می‌شود،
        به صورتی که از دو نقطه، که هر دو در ارتفاع h از سطح افق قرار دارند،
        عبور می‌کند.
        اگر پرتاب‌کننده طوری تعیین شده باشد که بیشترین برد (range) را داشته باشد،
        آنگاه نشان دهید در این حالت فاصله‌ی بین نقاط به صورت زیر نشان داده می‌شود:
        $$
            d = \frac{v_i}{g}\sqrt{v_i^2 - 4gh}
        $$
    \end{question}

    \vspace{3cm}
    مهلت ارسال: شنبه ۲۸ مهر، تا ساعت ۱۵:۱۲

\end{document}


