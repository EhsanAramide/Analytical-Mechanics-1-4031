% See `LICENCE' file in root directory for more information
% Only use XeLaTeX compiler 

\documentclass{article}

% Packages
\usepackage{fancyhdr}
\usepackage[a4paper, 
top=1in,
marginparwidth=0.6in,
left=1in, right=0.75in]
{geometry}
\usepackage{setspace}
\usepackage{xepersian}

% Packages initializations
% `xepersian'
\settextfont[Scale=1]{XB Zar}
\setlatintextfont[Scale=1]{Junicode}

% `fancyhdr'
\pagestyle{fancy}
\fancyhf[HR]{سید امیر رضا طالقانی / احسان آرمیده}
\fancyhf[HL]{مکانیک تحلیلی ۱ / تمرین سری ۱}
\fancyhf[FC]{\thepage}

% `setspace'
\linespread{2}

\begin{document}
\noindent
\section*{سوالات}
از فصل ۱ کتاب مکانیک تحلیلی (فولز ویراست ۷ام) سوالات روبه‌رو حل بشود:
۸ / ۱۴ / ۱۹ / ۲۳ / ۲۴
\\

\noindent
{\large سوال زیر نیز حل شود:}

\noindent
۱-
در حالت کلی مشتق زمانی اندازه بردار سرعت با اندازه بردار شتاب الزاما مساوی نمی باشد. به بیان دیگر اگر بردار سرعت و شتاب به ترتیب با $\vec{v}$ , $\vec{a}$ نمایش داده شوند و اندازه آنها با $v$ و $a$ , رابطه زیر الزاما برقرار نخواهد بود:
$$ \dot{v} = a $$
الف)\ با استفاده از مؤلفه های بردار های سرعت و شتاب در مختصات دکارتی , مقدار $\dot{v}$ را بدست اورده و با اندازه بردار شتاب $a$ مقایسه کنید. \\
ب)\ در چه حالتی رابطه $ \dot{v} = a $ از لحاظ ریاضی برقرار خواهد بود؟ \\
~\\
~\\
~\\
مهلت تحویل : شنبه 22 مهر
\end{document}


